\thispagestyle{plain}
\begin{center}
    \Large \textbf{\uppercase{Overview}}
\end{center}

\vspace{3\baselineskip}

\noindent
An unmanned aerial vehicle (UAV), commonly known as a drone, is an aircraft without any human pilot, crew, or passengers on board. UAVs are a component of an unmanned aircraft system (UAS), which has three basic components:
\begin{enumerate}
	\item An autonomous or human-operated control system which is usually on the ground or a ship but may be on another airborne platform.
	\item An Unmanned Aerial Vehicle (UAV)
	\item A command and control (C2) system - sometimes referred to as a communication, command, and control (C3) system - to link the two.
\end{enumerate}

\noindent These systems include but are not limited to remotely piloted air systems (RPAS) in which the UAV is controlled by a 'pilot' using a radio data link from a remote location. UAS can also include an autonomously controlled UAV or, more likely, a semi-autonomous UAV.

 \vspace{\baselineskip}
 
\noindent \textbf{NOTE}: In recent years, the tendency to refer to any UAV as a ``drone'' has developed but the term is not universally considered appropriate. UAVs can vary in size from those which can be hand launched to purpose-built or adapted vehicles the size of conventional fixed or rotary wing aircraft.

 \vspace{\baselineskip}

\noindent In this given task, we have designed and developed two individual UAS systems (Fixed-Wing sUAS and multi-rotor drone) with respect to the given configuration. The design and analysis of the UAS system are done with respect to the selection of parameters that suit the given configuration. A linearized dynamics model was developed and subjected to multiple simulations and test conditions in order to estimate the stability of the system. The control system is implemented by incorporating PID control over the control points of the aircraft and the test results are established. The software platform used for the analysis of the aircraft system is MATLAB and Simulink.

\vspace{\baselineskip}

\noindent
\textbf{Keywords}: sUAS, Multirotor, Control System, Fixed-wing
